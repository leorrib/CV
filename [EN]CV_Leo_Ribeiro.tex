%%%%%%%%%%%%%%%%%%%%%%%%%%%%%%%%%%%%%%%%%
% Developer CV
% LaTeX Template
% Version 1.0 (28/1/19)
%
% This template originates from:
% http://www.LaTeXTemplates.com
%
% Authors:
% Jan Vorisek (jan@vorisek.me)
% Based on a template by Jan Küster (info@jankuester.com)
% Modified for LaTeX Templates by Vel (vel@LaTeXTemplates.com)
%
% License:
% The MIT License (see included LICENSE file)
%
%%%%%%%%%%%%%%%%%%%%%%%%%%%%%%%%%%%%%%%%%

%----------------------------------------------------------------------------------------
%	PACKAGES AND OTHER DOCUMENT CONFIGURATIONS
%----------------------------------------------------------------------------------------

\documentclass[9pt]{developercv} % Default font size, values from 8-12pt are recommended
\usepackage{varwidth}
\usepackage{enumitem}
\usepackage{hyperref}
%----------------------------------------------------------------------------------------

\begin{document}

%----------------------------------------------------------------------------------------
%	TITLE AND CONTACT INFORMATION
%----------------------------------------------------------------------------------------

\begin{minipage}[t]{0.4\textwidth} % 45% of the page width for name
	\vspace{-\baselineskip} % Required for vertically aligning minipages
	
	% If your name is very short, use just one of the lines below
	% If your name is very long, reduce the font size or make the minipage wider and reduce the others proportionately
	%\colorbox{black}{{\HUGE\textcolor{white}{\textbf{\MakeUpp%ercase{Leonardo}}}}} % First name
	
	%\colorbox{black}{{\HUGE\textcolor{white}{\textbf{\MakeUpp%ercase{Ribeiro}}}}} % Last name
	
	%\vspace{6pt}
	
	{	
		%Curriculum Vitae\\\\ 
		\huge Leonardo Rodrigues Ribeiro \\\\
		\huge \textbf{Quality Assurance Analyst}
	} % Career or current job title
\end{minipage}
\begin{minipage}[t]{0.275\textwidth} % 27.5% of the page width for the first row of icons
	\vspace{-\baselineskip} % Required for vertically aligning minipages
	
	% The first parameter is the FontAwesome icon name, the second is the box size and the third is the text
	% Other icons can be found by referring to fontawesome.pdf (supplied with the template) and using the word after \fa in the command for the icon you want
	\icon{MapMarker}{12}{Tijuca, Rio de Janeiro}\\
	\icon{Phone}{12}{+55 (21) 96845-9688}\\
	%\icon{At}{12}{\href{mailto:alyx@vance.me}{alyx@vance.me}}\\	
\end{minipage}
\begin{minipage}[t]{0.300\textwidth} % 27.5% of the page width for the second row of icons
	\vspace{-\baselineskip} % Required for vertically aligning minipages
	
	% The first parameter is the FontAwesome icon name, the second is the box size and the third is the text
	% Other icons can be found by referring to fontawesome.pdf (supplied with the template) and using the word after \fa in the command for the icon you want
	\icon{At}{12}{\href{mailto:ribeiro.lrodrigues@gmail.com}{ribeiro.lrodrigues@gmail.com}}
	%\icon{Globe}{12}{\href{https://alyx.vance.me}{alyx.vance.me}}\\
	\icon{Github}{12}{\href{https://github.com/leorrib}{github.com/leorrib}}\\
	%\icon{Twitter}{12}{\href{https://twitter.com/@alyxvance}{@alyxvance}}\\
\end{minipage}

\vspace{0.5cm}

%----------------------------------------------------------------------------------------
%	INTRODUCTION, SKILLS AND TECHNOLOGIES
%----------------------------------------------------------------------------------------

%\begin{minipage}[t]{0.4\textwidth} % 40% of the page width for the introduction text
%	\vspace{-\baselineskip} % Required for vertically aligning minipages
	
%	\lorem \lorem \lorem \lorem \lorem\\ % Dummy text
%\end{minipage}
%\hfill % Whitespace between
\begin{minipage}[t]{0.5\textwidth} % 50% of the page for the skills bar chart
	\vspace{-\baselineskip} % Required for vertically aligning minipages
	\cvsect{Programming Languages}
	\begin{barchart}{5.5}
		\baritem{JavaScript}{90}
	    \baritem{R}{70}
		\baritem{Python}{70}
		\baritem{Java}{60}
		\baritem{Ruby}{50}
	\end{barchart}
\end{minipage}
\hfill % Whitespace between
\begin{minipage}[t]{0.53\textwidth} % 50% of the page for the skills bar chart
	\vspace{-\baselineskip} % Required for vertically aligning minipages
	\cvsect{Techs, Tools and Others}
		
		\begin{minipage}[t]{0.5\textwidth}
			\begin{itemize}[label={\large\textbullet}]
				\setlength\itemsep{0.01em}
				\item Selenium (Java \& Python)
				\item Junit \& TestNG 
				\item Rest Assured
				\item Supertest
				\item Cucumber \& Behave
				\item Protractor
			\end{itemize}
		\end{minipage}
		\begin{minipage}[t]{0.5\textwidth}
			\begin{itemize}
				\setlength\itemsep{0.01em}
				\item Azure Mach. Learn.
				\item Data Analysis (R)
				\item SQL
				\item Git (lab \& hub)
				\item Docker
				\item AWS basics
			\end{itemize}
	\end{minipage}
\end{minipage}
%

\cvsect{Who am I}

\ \ \ \ \ \ \ \ \ \ \ \ \ \ \ \ \ Data Scientist aspirant, I now work as QA Analyst at Globo. I have a peculiar Educational background: I own a PhD in Physics, obtained at the State University of Rio de Janeiro. I had the privilege of start working as a professor when I was only 25 and in one of the most relevant universities in Brazil, the State University of Rio de Janeiro. Afterwards, I was a Visiting Scholar at the University of Illinois at Urbana-Champaign and, at the age of 30, I was hired by the largest federal university in Brazil, the Federal University of Rio de Janeiro, where I taught a variety of Calculus subjects. At 32, after finishing my phd, I dedicated myself to study the subject of test automation and, today, I work as QA Analayst at Grupo Globo, the largest media company in Brazil.
%\begin{center}
%	\bubbles{
%		5/\begin{varwidth}[]{\linewidth}
%			\begin{center}
%				Selenium\\
%				WebDriver
%			\end{center}
%		\end{varwidth},
%		4/\begin{varwidth}[c]{\linewidth}
%			\begin{center}
%				Rest\\
%				Assured
%			\end{center}
%		\end{varwidth}, 
%		5/Protractor, 
%		5/Jenkins, 
%		5/Git, 
%		5/TestNG}
%\end{center}

%----------------------------------------------------------------------------------------
%	EXPERIENCE
%----------------------------------------------------------------------------------------
\cvsect{Professional Experience}

\begin{entrylist}
	\entry
	{2021 (current)}%\\\footnotesize{part time}}
	{Anl Quality Assurance II}
	{Globo Comunicação e Participações S/A}
	{Main responsibilities:\\
- Discussion/comprehension of business rules (with final consumers as well as with the development team).\\
- Definition of sets of criteria for a feature to be considered implemented.\\
- AWS log reading, mainly focused on bugs. That allows me to write a detailed reporting about the issues, making it easier and faster for developers to solve it.\\
- Creating API and UI automated tests (supertest/protractor).\\
- Database manipulation (MySQL).\\
- Maintenance of the integration between automated testes and the project's pipeline.\\
- Participation in the recruitment of new QA Analysts.\\
- Training new QA Analysts.}
	\entry
	{2020}%\\\footnotesize{part time}}
	{Test Analyst}
	{Noesis Consultoria e Programação de Sistemas de Informática LTDA}
	{Main responsabilities:\\
- Validating the continuous functionality of all the features of the software being developed.\\
- Writing automated API and UI tests.\\
- Data handling on databases (SQL).}
	\entry
	{2017 -- 2018}
	{Professor}
	{UFRJ}
	{- Subjects taught: Mathematics II, Calculus II and Calculus III.}
	\entry
	{2015 -- 2016}
	{Research Scholar}
	{University of Illinois at Urbana-Champaign}
	{- Research in Condensed Matter Physics (multidimensional bosonization of systems under a weak magnetic field).}
	\entry
	{2013 -- 2014}
	{Professor}
	{UERJ}
	{- Subjects taught: Mathematics Applied to Business, Calculus I, Calculus IV and Complex Analysis.}
\end{entrylist}

%----------------------------------------------------------------------------------------
%	EDUCATION
%----------------------------------------------------------------------------------------
\newpage
\cvsect{Academic Background}

\begin{entrylist}
	\entry
	{2013 -- 2019}
	{PhD (Physics)}
	{UERJ}
	{- Thesis title: A constructive approach to the Bosonization of Fermi Liquids subjected to a homogeneous and weak magnetic field.\\ 
		- Eighteen months spent as a Research Scholar at theUniversity of Illinois Urbana-Champaign.\\
		- Scholarship received from Ci\^encia Sem Fronteiras.\\
		- Scholarship received from Lehman Foundation.\\
		- Scholarship received from Coordenação de Aperfeiçoamento de Pessoal de Nível Superior (CAPES).}
	\entry
	{2011 -- 2013}
	{Masters (Physics)}
	{UERJ}
	{- Thesis title: Transições de Fase e Interações Competitivas.\\
		- Scholarship received from Conselho Nacional de Pesquisa e Desenvolvimento (CNPQ).}
	\entry
	{2005 -- 2010}
	{Bachelor's degree (Physics)}
	{UERJ}
	{- Thesis title: Supercondutividade: A equação de London e o Modelo de Ginzburg-Landau.\\
		- Scholarship received from Fundação Carlos Chagas Filho de Amparo à Pesquisa do Estado do Rio de  Janeiro (FAPERJ).}
\end{entrylist}

%----------------------------------------------------------------------------------------
%	TECH EDUCATION
%----------------------------------------------------------------------------------------
\cvsect{Tech Education}

\begin{entrylist}
	\entry
		{2021}%\\\footnotesize{part time}}
		{Big Data Analytics with R and Microsoft Azure Machine Learning}
		{DataScienceAcademy}
		{Discusses the main features of the R language used in Data Science. Besides, it also introduces data visualization and manipulation techniques, aside from analysis and the construction of Machine Learning Models using R and Microsoft Azure Machine Learning.\\ 
		\texttt{Data Analysis}\slashsep\texttt{Machine Learning}\slashsep\texttt{Microsoft Azure Machine Learning}}
	\entry
		{2020}%\\\footnotesize{part time}}
		{SQL Fundamentals}
		{Datacamp}
		{Discusses fundamental skills needed to interact with and query data in SQL.\\ 
		\texttt{SQL}\slashsep\texttt{database}}
	\entry
		{2020}
		{Learn to Code with Python}
		{Udemy}
		{Introduces from core language fundamentals to advanced features like classes, decorators and unit testing. The final project is the development of a poker game.\\ 
		\texttt{Python}\slashsep\texttt{unit testing}}
	\entry
		{2020}%\\\footnotesize{part time}}
		{Python Fundamentals}
		{Datacamp}
		{Python basics, including how to clean real-world data ready for analysis and use data visualization libraries.\\ 
		\texttt{Numpy}\slashsep\texttt{Matplotlib}\slashsep\texttt{Pandas}}
	\entry
		{2020}
		{SDET/Test Architect Essentials - Road to Full stack QA}
		{Udemy}
		{Introduces the implementation of Docker and discusses Jenkins Pipelines.\\ 
		\texttt{Docker-Selenium}\slashsep\texttt{Jenkins Pipelines}}
	\entry
		{2020}
		{Learn Protractor (Angular Testing) from scratch + Framework}
		{Udemy}
		{Discusses the creation of User Interface automated tests using protractor.\\
		\texttt{Protractor}\slashsep\texttt{Javascript}\slashsep\texttt{Typescript}}
	\entry
		{2020}% \\\footnotesize{(em andamento)} }
		{NodeJS Rest - ExpressJS Mongodb - Jest Unit/Int Test - 2020}
		{Udemy}
		{Develops a Rest API by using NodeJS/ExpressJS. Moreover, it also teaches how to create unit and integration tests.\\
		\texttt{NodeJS}\slashsep\texttt{ExpressJS}\slashsep\texttt{Jest}\slashsep\texttt{MongoDB}\slashsep\texttt{Supertest}}	
	\entry
		{2020}%\\\footnotesize{part time}}
		{Selenium WebDriver with Java - Basics to Advanced + Frameworks}
		{Udemy}
		{Discusses the creation of User Interface automated tests using Selenium Webdriver (with Java).\\ 
		\texttt{Selenium WebDriver}\slashsep\texttt{Maven}\slashsep\texttt{TestNG}\slashsep\texttt{Cucumber}}
	\entry
		{2020}% \\\footnotesize{(em andamento)} }
		{Rest API Testing (Automation) from Scratch - RestAssured Java}
		{Udemy}
		{Discusses the creation of API integration tests using RestAssured.\\
		\texttt{Postman}\slashsep\texttt{Rest Assured}\slashsep\texttt{Java}}
	\entry
		{2020}%\\\footnotesize{part time}}
		{Selenium Python Automation Testing from Scratch + Frameworks}
		{Udemy}
		{Discusses the creation of User Interface automated tests using Selenium Webdriver (with Python).\\ 
		\texttt{Selenium}\slashsep\texttt{Pytest}}
\end{entrylist}



%--------------------------------------------------------------------------
% Artigos Publicados
% ----------------------------------------------------------------------
\newpage
\cvsect{Papers Published in Scientific Journals}

\begin{entrylist}
	\entry
	{2018}
	{BARCI, DANIEL G.; FRADKIN, EDUARDO; RIBEIRO, LEONARDO}
	{Physical Review B}
	{- Title: Bosonization of Fermi liquids in a weak magnetic field.\\
	- Reference: Physical Review B v. 98, p. 155146, 2018.\\
	- arXiv: \href{https://arxiv.org/abs/1805.05337}{arxiv.org/abs/1805.05337} }
	\entry
	{2013}
	{BARCI, DANIEL G.; RIBEIRO, LEONARDO; STARIOLO, DANIEL A.}
	{Physical Review E}
	{- Title: Nematic phase in two-dimensional frustrated systems with power-law decaying interactions.\\
	- Reference: Physical Review E (Statistical, Nonlinear, and Soft Matter Physics), v.87, 062119.\\
	- arXiv: \href{http://arxiv.org/abs/1306.1204}{arxiv.org/abs/1306.1204} }
\end{entrylist}

%----------------------------------------------------------------------------------------
%	ADDITIONAL INFORMATION
%----------------------------------------------------------------------------------------

\begin{minipage}[t]{0.3\textwidth}
	\vspace{-\baselineskip} % Required for vertically aligning minipages

	\cvsect{Languages}
	
	\textbf{Portuguese} - native\\
	\textbf{English} - fluent\\
	\textbf{Spanish} - average
\end{minipage}
\hfill
\begin{minipage}[t]{0.3\textwidth}
	\vspace{-\baselineskip} % Required for vertically aligning minipages
	
	\cvsect{Trabalho voluntário}
	
	Ex-professor no pré-vestibular comunitário da Mangueira.
\end{minipage}

%----------------------------------------------------------------------------------------

\end{document}

\hfill
\begin{minipage}[t]{0.3\textwidth}
	\vspace{-\baselineskip} % Required for vertically aligning minipages
	
	\cvsect{Hobbies}
	
	Karaokê (péssimo cantor, mas com orgulho) e futebol.
\end{minipage}
